
\section{Good numbers to know}

\Table{
\hline

Mass of proton & $1.67 \times 10^{-27}$ kg

\\ \hline

Diameter of proton & about a femtometer $(10^{-15})$
 
\\ \hline

Diameter of an atom & about 0.1 nanometers = 1 \angst

\\ \hline

Mass of electron & $9.1 \times 10^{-31}$ kg
 
\\ \hline

Elementary charge & $1.6 \times 10^{-19}$ C

\\ \hline

Mass of the Earth & $6 \times 10^{24}$ kg

\\ \hline

Radius of the Earth & $6 \times 10^6$ m

\\ \hline

Mass of the Sun & $2 \times 10^{30}$ kg

\\ \hline

Radius of the Sun & 10,000 km = $7 \times 10^8 $ m

\\ \hline

The astronomical unit & 150 million km = $1.5 \times 10^{11} $ m

\\ \hline

Density of water & 1000 kg/m$^3$

 \\ \hline

}

%===========================================%

\Table{
\hline
\MiniPg{.2}{
Wavelength frequency ranges of visible light. 
}
&
\MiniPg{.6}{
\GraphicWHN{.6}{.4}{color_wavelength_frequency.png}

\tiny \url{http://reikiservices.cfsites.org/files/color_wavelength_frequency.png}
}
 \\ \hline

}

%===========================================%

\GraphicWHN{.9}{.5}{EM_Spectrum.png}
\center
\tiny \url{https://xkcd.com/273/} \large
\flushleft

%%%%%%%%%%%%%%%%%%%%%%%%%%%%%%%%%
%===========================================%
%%%%%%%%%%%%%%%%%%%%%%%%%%%%%%%%%



\section{Math}

\Table{
\hline

\MiniPg{.3}{
Normal Gaussian Distribution
}
&

\MiniPg{.7}{
\center

$f(x) = \dfrac{1}{\sqrt{2 \sigma^2 \pi}} e^{- \frac{(x- \mu)^2}{2 \sigma^2}}$,

where $\mu$ is expectation, $\sigma$ is standard deviation, $\sigma^2$ is variance.

}

\\ \hline

Empirical Rule

&

\MiniPg{.7}{

\GraphicWHN{1}{.57}{Empirical_Rule.PNG}.
\tiny Stolen without permission from the internet.
}

\\ \hline
}


%===========================================%


\subsection{Trigonometry}
\center

\GraphicWHN{.6}{.6}{UnitCircle.png}
\center \tiny \url{https://en.wikipedia.org/wiki/Unit_circle} \large

\Table{
\hline
\MiniPg{.3}{\center
$\dfrac{\sqrt{2}}{2}$
}
&
\MiniPg{.7}{\center
$\dfrac{\sqrt{2}}{2} \approx \dfrac{1.4}{2} = 0.7$
}
\\ \hline

$\dfrac{\sqrt{3}}{2}$

&

$\dfrac{\sqrt{3}}{2} \approx \dfrac{1.7}{2} = .85$

\\ \hline
}

%%%%

\Table{
\hline

First three pythagorean triples

&
\MiniPg{.6}{\center
$(3, 4, 5), \ \ \ (5, 12, 13), \ \ \ (8, 15, 17)$
}

\\ \hline
}

%%%%

\Table{
\hline

$\sin(\theta \pm \phi) = $ 
&
\MiniPg{.7}{ \center
$\sin(\theta)\cos(\phi) \pm \cos(\theta) \sin(\phi)$
}

\\ \hline

$\cos(\theta \pm \phi) = $ & $ \cos(\theta) \cos(\phi) \mp \sin(\theta) \sin(\phi)$

\\ \hline

$\cos(\theta) \cos(\phi) = $ & $  \dfrac{1}{2}[\cos(\theta + \phi) + \cos(\theta - \phi)]$

\\ \hline

$\sin(\theta) \sin(\phi) = $ & $  \dfrac{1}{2}[\cos(\theta - \phi) - \cos(\theta + \phi)]$

\\ \hline

$\sin(\theta) \cos(\phi) = $ & $ \dfrac{1}{2}[\sin(\theta + \phi) + \sin(\theta - \phi)]$

\\ \hline

}

%%%%

\Table{
\hline

$\sin(\theta) + \sin(\phi) = $ 

& 

\MiniPg{.7}{
\center 
$2 \sin\Big(\dfrac{\theta + \phi}{2} \Big) \, \cos \Big(\dfrac{\theta - \phi}{2} \Big) $
}

\\ \hline

$\sin(\theta) - \sin(\phi) = $ & $2 \sin\Big(\dfrac{\theta - \phi}{2} \Big) \, \cos \Big(\dfrac{\theta + \phi}{2} \Big) $

\\ \hline

$\cos(\theta) + \cos(\phi) = $ & $2 \cos\Big(\dfrac{\theta - \phi}{2} \Big) \, \cos \Big(\dfrac{\theta + \phi}{2} \Big) $

\\ \hline

$\cos(\theta) - \cos(\phi) = $ & $- 2 \sin\Big(\dfrac{\theta - \phi}{2} \Big) \, \sin \Big(\dfrac{\theta + \phi}{2} \Big) $

\\ \hline
}


\Table{
\hline
$\cos^2(\theta) = $ & 
\MiniPg{.7}{\center
$\dfrac{1}{2}[1 + \cos(2\theta)]$
}

\\ \hline

$\sin^2 (\theta) = $ & $ \dfrac{1}{2}[1-\cos(2 \theta)]$

\\ \hline

$\cos^2(\theta) + \sin^2(\theta) = $ & $ 1$

\\ \hline

$e^{i \theta} = $ & $ \cos \theta + i \sin \theta$

\\ \hline

$\cos \theta = $ (Euler's) & $ \dfrac{1}{2}(e^{i \theta} + e^{-i \theta}) $

\\ \hline

$\sin \theta = $ (Euler's) & $ \dfrac{1}{2i}(e^{i \theta} - e^{-i \theta}) $

\\ \hline
}


%===========================================%


\subsection{Calculus}

\Table{
\hline

\MiniPg{.3}{
The first fundamental theorem of calculus...}
&
\MiniPg{.7}{ ... states that if $f$ is continuous on the
closed interval $[a,b]$ and $F$ is the indefinite integral of $f$ on $[a,b]$ then
$\int^b_a f(x) dx = F(b) - F(a)$.}

\\ \hline

\MiniPg{.3}{
The second fundamental theorem of calculus ... }
&
\MiniPg{.7}{... holds for $f$, a continuous
function on an open interval $l$, and $a$, any point in $l$, and states that if $F$ is defined by the integral (antiderivative) $F(x) = \int^x_a f(x) dt$, then $F'(x) = f(x)$ at each point in $l$, where $F'(x)$ is the derivative of $F(x)$.}

\\ \hline

If $f=f(x,y)$ then $df = $ & $\dfrac{\partial f}{\partial x}dx + \dfrac{\partial f}{\partial y}dy$

\\ \hline
}

\Table{
\hline

Chain rule &
\MiniPg{.6}{

If we have z(y) and y(x), then $\dfrac{dz}{dx} = \dfrac{dz}{dy} \cdot \dfrac{dy}{dx}$

}

\\ \hline
}


%===========================================%


\subsection{Taylor and Maclaurin series}

\Table{
\hline
\MiniPg{.3}{
\center
$f(x) = $

Expand about the point $a$.}
&
\MiniPg{.7}{\center
$ f(a) + f'(a)(x-a) + \dfrac{1}{2!}f''(a)(x-a)^2 +\, \dfrac{1}{3!}f'''(a)(x-a)^3 ...$
}
\\ \hline

$e^x = $ & $1 + x + \dfrac{1}{2!}x^2 + \dfrac{1}{3!}x^3 + ...$

\\ \hline

$(1+x)^n = $ & $ 1 + nx + \dfrac{n(n-1)}{2!}x^2 + ... [|x|<1]$ [binomial series]

\\ \hline

$\cos(x) = $ & $ 1 - \dfrac{1}{2!}x^2 + \dfrac{1}{4!}x^4 + ... $

\\ \hline

$\cosh(x) = $ & $ 1 + \dfrac{1}{2!}x^2 + \dfrac{1}{4!}x^4 + ...$

\\ \hline
}

%%%%

\Table{
\hline

$\sin(x) = $
&
\MiniPg{.7}{
\center

$x - \dfrac{1}{3!}x^3 + \dfrac{1}{5!}x^5 - ...$

Also, for small angles, $\sin \theta \approx \tan \theta$.

} 
 \\ \hline

$\sinh(x) = $ & $x + \dfrac{1}{3!}x^3 + \dfrac{1}{5!}x^5 + ...$

\\ \hline

$\tan(x) = $ & $x + \dfrac{1}{3!}x^3 + \dfrac{1}{15!}x^5 + ... [|x|<\pi / 2]$

\\ \hline

$\tanh(x) = $ & $x - \dfrac{1}{3!}x^3 + \dfrac{1}{15!}x^5 - ... [|x|<\pi / 2]$

\\ \hline

$\ln(1 + x) = $ & $x - \dfrac{1}{2}x^2 +\dfrac{1}{3}x^3 - ... [|x| < 1]$

\\ \hline


}

%===========================================%

\subsection{Vector Calculus}  

\Table{
\hline

Laplacian &

\MiniPg{.5}{\center
 $\nabla^2 T = \vv{\nabla} \cdot (\vv{\nabla}T) = \dfrac{\partial^2T}{\partial x^2} + \dfrac{\partial^2T}{\partial y^2} +\dfrac{\partial^2T}{\partial z^2}$

$\nabla^2\vv{v} = \nabla^2v_x \hat x + \nabla^2v_y \hat y +\nabla^2v_z \hat z$
}

\\ \hline

\MiniPg{.4}{\center
What are the two zero vector derivatives? }
&
\MiniPg{.5}{\center
 $\vv{\nabla} \times(\vv{\nabla}T) = 0 $ 

$\vv{\nabla} \cdot (\vv{\nabla} \times \vv{v}) = 0$
}
\\ \hline

$\vv{\nabla} \times (\vv{\nabla} \times \vv{v} ) $ =  & $\vv{ \nabla}(\vv{\nabla} \cdot \vv{v} ) - \nabla^2 \vv{v} $

\\ \hline

Gradient Theorem: 	&
\MiniPg{.5}{\center
 $ \int_a^b (\vv{\nabla}f) \cdot d\vv{l} = f(b) - f(a) $  
 
  Independent of Path. 
  
  $\oint (\vv{\nabla}f) \cdot d\vv{l} = 0 $
}
\\ \hline

Stoke's Theorem: & $ \int_{surface} (\vv{\nabla} \times \vv{v}) \cdot d\vv{a} = \oint_{path} \vv{v} \cdot d\vv{l} $

\\ \hline

Green's (Divergence) Theorem: & $ \int_{volume}(\vv{\nabla} \cdot \vv{v}) d\tau = \oint_{surface} \vv{v} \cdot d\vv{a}$

\\ \hline

}


%%%%%%%%%%%%%%%%%%%%%%%%%%%%%%%%%%%%%%%%%%%%

\subsection{Matrices}

\Table{
\hline

\MiniPg{.3}{
Typical eigenvalue problem
}
&

\MiniPg{.7}{
\center

\MPalign{
\hat{T} \ket{\alpha} &= \lambda \ket{\alpha} \\
\Mtx{T}\Mtx{a} &= \lambda \Mtx{a} \\
(\Mtx{T} - \lambda \Mtx{I})\Mtx{a} &= \Mtx{0} 
}

\flushleft
By assumption, $\Mtx{a} \neq \Mtx{0}$, so $(\Mtx{T} - \lambda \Mtx{I})$ is singular: $\textrm{det}(\Mtx{T} - \lambda \Mtx{I}) = 0$. This gives an algebraic equation for $\lambda$, which is called the characteristic equation for the matrix; its solutions determine the eigenvalues:

\center
$C_n\lambda^n + C_{n-1}\lambda^{n-1} + ... + C_1 \lambda + C_0 = 0$,

\flushleft
where $n$ is the dimensionality of the vector space. By what's apparently called the `fundamental theorem of algebra', there are $n$ complex roots (the eigenvalues). Well, for an $n\times n$ matrix, there is at least one and at most $n$ distinct eigenvalues. The collection of the eigenvalues of a matrix is called its spectrum. If two or more linearly independent eigenvectors share the same eigenvalue, the spectrum is degenerate. To construct the eigenvectors, plug each $\lambda$ back into $\Mtx{T}\Mtx{a} = \lambda \Mtx{a}$ and solve by one-by-one. \tiny This entry paraphrased and largely quoted from Griffiths, \textit{Introduction to Quantum Mechanics} Appendix 5.

}

\\ \hline
}

%%%%

\Table{
\hline

Diagonalization

&

\MiniPg{.7}{
\center

If the eigenvectors span the space, they can be used as a basis.

$\hat{T} \ket{f_n} = \lambda \ket{f_n}$,

$\Mtx{T} =
\SmallMatrix{
\lambda_1 & 0 & ... & 0 \\
0 & \lambda_2 & ... & 0 \\
\vdots & \vdots & .\vdots. & \vdots \\
0 & 0 & ... & \lambda_n
}$

and the normalized eigenvectors are

$\SmallMatrix{1 \\0 \\ \vdots \\ 0}, \SmallMatrix{0 \\ 1 \\ \vdots \\ 0}, ...,\SmallMatrix{0 \\0 \\ \vdots \\ 1}$.

A matrix that can be brought to diagonal form by a change of basis is said to be diagonalizable. Also note that the trace of a matrix (sum of the diagonal components) is invariant under transformation.

 \tiny This entry paraphrased and largely quoted from Griffiths, \textit{Introduction to Quantum Mechanics} Appendix 5.
}

\\ \hline
}






























